\documentclass[11pt,a4paper]{jsarticle}
%
\usepackage{amsmath,amssymb}
\usepackage{bm}
\usepackage{graphicx}
\usepackage{float}
\usepackage{ascmac}
%
\setlength{\textwidth}{\fullwidth}
\setlength{\textheight}{39\baselineskip}
\addtolength{\textheight}{\topskip}
\setlength{\voffset}{-0.5in}
\setlength{\headsep}{0.3in}
%
%\setlength{\topmargin} {1pt} % = 0.3528mm
\setlength{\textheight}{200mm} 
%
\newcommand{\divergence}{\mathrm{div}\,}  %ダイバージェンス
\newcommand{\grad}{\mathrm{grad}\,}  %グラディエント
\newcommand{\rot}{\mathrm{rot}\,}  %ローテーション
%
\pagestyle{myheadings}
\markright{\footnotesize \sf BL04解析  TOF-Energy 変換}
\begin{document}
%


\begin{eqnarray}
\Delta \sigma _{cp}= \kappa (J) \frac{w}{v}\Delta \sigma_{p}
\nonumber
\end{eqnarray}




\begin{eqnarray}
\sigma(E)=\frac{P1}{\sqrt{E}}\frac{P2^{2}}{(E-P3)^2-P2^2/4}+\frac{P4}{\sqrt{E}}\frac{P5^{2}}{(E-P6)^2-P5^2/4}
\nonumber
\end{eqnarray}




\begin{eqnarray}
P(JJ'jj'kIF)=(-1)^{J+J'+j'+I+F}\frac{3}{2}\sqrt{(2J+1)(2J'+1)(2j+1)(2j'+1))}
\left\{
\begin{tabular}{ccc}
$k$ & $ j$ & $j' $\\
$I$ & $J'$ & $J $\\
\end{tabular}
\right\} \left\{
\begin{tabular}{ccc}
$k$ &  1 & 1  \\
$F$ &  $J$ & $J'$  \\
\end{tabular}
\right\}   \nonumber
\end{eqnarray}

\begin{eqnarray}
| \kappa(J) |
\end{eqnarray}

\begin{eqnarray}
\Gamma_{D} = \sqrt{\frac{4E_0 k T}{A}}
\end{eqnarray}









%a_9
$a_9$


\begin{eqnarray}
\frac{d\sigma}{d\Omega} &=& \frac{1}{2}  ( a_0+ a_1 cos \theta_\gamma + a_3\left(cos^2 \theta_\gamma -\frac{1}{3} \right)  \nonumber \\ 
&+&a_9({\bf \sigma}_n \cdot {\bf k}_\gamma) + a_{12}\left(({\bf \sigma}_n \cdot {\bf k}_n)({\bf k}_n \cdot {\bf k}_\gamma) -\frac{1}{3}({\bf \sigma}_n \cdot {\bf k}_\gamma) \right)
)
\end{eqnarray}

\begin{eqnarray}
\left(\frac{d\sigma}{d\Omega}\right)_\uparrow - \left(\frac{d\sigma}{d\Omega}\right)_\downarrow
=a_9-\frac13a_{12}
\nonumber
\end{eqnarray}


\begin{eqnarray}
 a_9=
 -2Re(\sum_{J_s,J'_s} V_1(J_s)V^*_3(J'_s)P(J_sJ'_s \frac12 \frac12 1 I F) \nonumber \\
 +\sum_{J_p,j,J'_p,j'} V_2(J_pj)V^*_4(J'_pj')P(J_pJ'_p j j' 1 I F)  \nonumber \\
\times 6 \left\{
\begin{array}{ccc}
0&1&1\\
1&\frac12&\frac12\\
1&j&j'
\end{array}
\right\}
 )
  \end{eqnarray}

\begin{eqnarray}
 a_{12}=
 -2Re\sum_{J_s,j,J'_p,j'} V_2(J_pj)V^*_4(J'_pj')P(J_pJ'_p j j' 1 I F) 
\times 18 \left\{
\begin{array}{ccc}
2&1&1\\
1&\frac12&\frac12\\
1&j&j'
\end{array}
\right\}
 )
  \end{eqnarray}

%a_10
$a_{10}$

\begin{eqnarray}
\frac{d\sigma}{d\Omega} &=& \frac{1}{2}  ( a_0+ a_1 cos \theta_\gamma + a_3\left(cos^2 \theta_\gamma -\frac{1}{3} \right)  \nonumber \\ 
&+&a_{10}({\bf \sigma}_n \cdot {\bf k}_n) + a_{11}\left(({\bf \sigma}_n \cdot {\bf k}_\gamma)({\bf k}_n \cdot {\bf k}_\gamma) -\frac{1}{3}({\bf \sigma}_n \cdot {\bf k}_n) \right)
)
\end{eqnarray}

\begin{eqnarray}
\left(\frac{d\sigma}{d\Omega}\right)_+ - \left(\frac{d\sigma}{d\Omega}\right)_-
=a_{10}-\frac13a_{11}
\nonumber
\end{eqnarray}


\begin{eqnarray}
 a_{10}=
 -2Re\sum_{J_s} \left(V_2(J_p=J_s,1/2)V^*_3(J_s)
 +V_1(J_s)V^*_4(J_p=J_s,1/2) \right) \nonumber \\
  \end{eqnarray}

\begin{eqnarray}
 a_{11}=
 2Re\sum_{J_s,J_p} V_2(J_p,\frac32)V^*_3(J_s) +V_1(J_s)V^*_4(J_p,\frac32) \sqrt{3}P(J_sJ_p \frac12 \frac32 2 I F) 
  \end{eqnarray}

%T-violationの散乱振幅

\begin{eqnarray}
f^{'}=A^{'} + B^{'} {\pmb \sigma} \cdot \hat{I} +C ^{'}{\pmb \sigma} \cdot \hat{k}+D^{'} {\pmb \sigma} \cdot (\hat{I} \times \hat {k})   \nonumber \\
f=A + B {\pmb \sigma} \cdot \hat{I} +C{\pmb \sigma} \cdot \hat{k}+D {\pmb \sigma} \cdot (\hat{I} \times \hat {k})    \nonumber 
\end{eqnarray}


\begin{eqnarray}
A=e^{iZA^{'}} \cos b \nonumber\\
B=e^{iZA^{'}} \frac{\sin b}{b}ZB^{'} \nonumber\\
C=e^{iZA^{'}} \frac{\sin b}{b}ZC^{'} \nonumber\\
D=e^{iZA^{'}} \frac{\sin b}{b}ZD^{'} \nonumber\\
Z=2\pi\rho z/k  \nonumber\\
b=Z\sqrt{B^{'2}+C^{'2}+D^{'2}}
\end{eqnarray}

\begin{eqnarray}
A=e^{iZA^{'}} \cos b , \hspace{5mm}
B=e^{iZA^{'}} \frac{\sin b}{b}ZB^{'} ,  \hspace{5mm} 
C=e^{iZA^{'}} \frac{\sin b}{b}ZC^{'} \nonumber\\
D=e^{iZA^{'}} \frac{\sin b}{b}ZD^{'} ,   \hspace{5mm}
Z=2\pi\rho z/k  ,   \hspace{5mm}
b=Z\sqrt{B^{'2}+C^{'2}+D^{'2}}
\end{eqnarray}


\begin{eqnarray}
\omega & = & 2\pi\rho B^{'}v/k \\
 & = & \gamma B^{'}_{pseudo}
\end{eqnarray}




\begin{eqnarray}
cos\theta
\end{eqnarray}



\begin{eqnarray}
\left(\frac{d\sigma}{d\Omega}\right)_{\theta=70^{\circ}} - \left(\frac{d\sigma}{d\Omega}\right)_{\theta=110^{\circ}}
= a_{1} cos 70^\circ
\end{eqnarray}




\begin{eqnarray}
\left(\frac{d\sigma}{d\Omega}\right)_{\theta=70^{\circ}} + \left(\frac{d\sigma}{d\Omega}\right)_{\theta=110^{\circ}}
-2\left(\frac{d\sigma}{d\Omega}\right)_{\theta=90^{\circ}}
= a_{3}\left(2 cos^2 70^\circ  - \frac{4}{3} \right)
\end{eqnarray}



\begin{eqnarray}
A_1=\frac{a_{1x}cos\phi+a_{1y}sin\phi}{a0}
\end{eqnarray}

\begin{eqnarray}
A_3=\frac{a_{3xy}cos\phi\sin\phi+a_{3yy}sin^2\phi}{a0}
\end{eqnarray}



\begin{eqnarray}
\vec{\sigma}\cdot (\vec{k}\times \vec{I})
\end{eqnarray}

\begin{eqnarray}
\vec{\sigma}\cdot \vec{k}\times \vec{I}\cdot \vec{r}
\end{eqnarray}

\begin{eqnarray}
\vec{\sigma} = \vec{r} \times \vec{k}
\end{eqnarray}

\begin{eqnarray}
h = \vec{\sigma} \cdot \vec{k}
\end{eqnarray}

\begin{eqnarray}
{
j = l + s}
\end{eqnarray}





\begin{eqnarray}
A_L = \frac{\sigma^{+} - \sigma^{-}}{\sigma^{+} + \sigma^{-}}
\end{eqnarray}






\begin{eqnarray}
\frac{d\sigma}{d\Omega} 
&= & \frac{1}{2}(a_0+a_1{\bf k}_n\cdot {\bf k}_\gamma
+a_3(({\bf k}_n\cdot {\bf k}_\gamma)^2-\frac{1}{3}))   \nonumber \\
%&= &\frac{1}{2}(a_0+a_1 cos\theta+a_3(cos^2 \theta-\frac{1}{3})     \nonumber \\
&= &\frac{a_0}{2}(1+A_1 cos\theta
+A_3(cos^2 \theta-\frac{1}{3})   \nonumber
\end{eqnarray}

\hspace{3cm}


\begin{eqnarray}
a_1=2 Re\sum_{J_s,J_p,j}^{ }  V_1(J_s)V_2^*(J_p j)P(J_s J_p \frac{1}{2}j 1 I F)
\end{eqnarray}



\begin{eqnarray}
V_1(J_s)= \frac{1}{2k_s}\sqrt{\frac{E_s}{E}}\frac{\sqrt {g\Gamma_s^n \Gamma_\gamma}}{E-E_s +i \Gamma_s/2}   \nonumber \\
V_2(J_pj)= \frac{1}{2k_p}\sqrt{\frac{E_p}{E}}\sqrt{\frac{\Gamma_{pj}^{n}}{\Gamma^n_p}}\frac{\sqrt {g\Gamma_p^n \Gamma_\gamma}}{E-E_p +i \Gamma_p/2}
\end{eqnarray}







\begin{eqnarray}
\frac{d\sigma}{d\Omega} 
&= &\frac{a_0}{2}(1+A_1 cos\theta
+A_3(cos^2 \theta-\frac{1}{3})   \nonumber
\end{eqnarray}








\begin{eqnarray}
a_1=a_{1x}cos\phi+a_{1y}sin\phi  \\
\\
a_3=a_{3xy}cos\phi sin\phi+a_{3yy}sin^2\phi \\
\\
\frac{n_L-n_H}{n_L+n_H}
\end{eqnarray}






\hspace{3cm}

$^{139}La$ $E_n=0.734eV$に対して
$$
(A_1)_{E=E_p\pm\Gamma_p/2} \simeq \pm (-0.2194x+0.2595y)  
$$
$$
(A_3)_{E=E_p} \simeq -0.4258xy+0.07197y^2
$$

\hspace{3cm}
A$_1$の値はE=E$_p$のp波共鳴点の前後で符号をかえる。\\
\hspace{3cm}
そこでE=E$_p$より高いエネルギー領域の断面積をN$_f$、\\
\hspace{3cm}
低いエネルギー領域をN$_b$とすると、
$$
N_f=C(1+A_1 cos\theta + A_3 (cos^2 \theta-\frac{1}{3}))
$$
$$
N_b=C(1-A_1 cos\theta + A_3 (cos^2 \theta-\frac{1}{3}))
$$

\hspace{3cm}
この式から
\begin{eqnarray}
\frac{N_f+N_b}{2} &=&C(1+A_3(cos^2\theta-\frac{1}{3})  \nonumber \\
\frac{N_f-N_b}{2}&=&CA_1cos\theta \nonumber
\end{eqnarray}

$$
\frac{R_f-R_b}{RS}=-0.007
$$

\newpage
$$
(A_1)_{E=E_p+\Gamma_p/2} \simeq -0.2194x+0.2595y  
$$
$$
(A_3)_{E=E_p} \simeq -0.4258xy+0.07197y^2
$$
\hspace{4cm}において
$$
x=cos\phi
$$
$$
y=sin\phi
$$
\hspace{4cm}として、実験より求まったA$_1$,A$_3$と比較


この式から
\begin{eqnarray}
x^2 = cos^2\phi = \Gamma^n_{p,1/2}/\Gamma^n_p  \nonumber \\
y^2 = sin^2\phi = \Gamma^n_{p,3/2}/\Gamma^n_p \nonumber  \\
x^2+y^2=1 \nonumber
\end{eqnarray}



\begin{eqnarray}
\kappa (J=I+\frac{1}{2})=\frac{3}{2\sqrt{2}}\left( \frac{2I+1}{2I+3} \right)
\frac{\sqrt{2I+1}(2\sqrt{I}x-\sqrt{2I+3}y)}{(2I-3)\sqrt{2I+3}x-(2I+9)\sqrt{I}y}
\nonumber\\
\kappa (J=I-\frac{1}{2})=\frac{3}{2\sqrt{2}}\left( \frac{(2I+1)\sqrt{I}}{\sqrt{(I+1)(2I-1)}} \right)
\frac{2\sqrt{I+1}x+\sqrt{2I-1}y}{(I+3)\sqrt{2I-1}x+(4I-3)\sqrt{I+1}y}
\nonumber\\
\end{eqnarray}





%
%
\end{document}
